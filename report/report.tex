\documentclass[conference]{IEEEtran}
\usepackage{cite}
\usepackage{amsmath,amssymb,amsfonts}
\usepackage{algorithmic}
\usepackage{graphicx}
\usepackage{textcomp}
\usepackage{xcolor}
\def\BibTeX{{\rm B\kern-.05em{\sc i\kern-.025em b}\kern-.08em
    T\kern-.1667em\lower.7ex\hbox{E}\kern-.125emX}}
\begin{document}

% Your paper title should be specific, concise, and descriptive.
% Avoid using unnecessary words such as “new” or “novel”.
% Include keywords that will help a reader find your paper.
\title{Conference Paper Title*\\
}

\makeatletter
\newcommand{\linebreakand}{
    \end{@IEEEauthorhalign}
    \hfill\mbox{}\par
    \mbox{}\hfill\begin{@IEEEauthorhalign}
}
\makeatother

\author{\IEEEauthorblockN{1\textsuperscript{st} Given Name Surname}
    \IEEEauthorblockA{\textit{College of Engineering (of Aff.)} \\
        \textit{Boston University (of Aff.)}\\
        Boston, USA \\
        email address or ORCID}
    \and
    \IEEEauthorblockN{2\textsuperscript{nd} Given Name Surname}
    \IEEEauthorblockA{\textit{College of Engineering (of Aff.)} \\
        \textit{Boston University (of Aff.)}\\
        Boston, USA \\
        email address or ORCID}
    \and
    \IEEEauthorblockN{3\textsuperscript{rd} Given Name Surname}
    \IEEEauthorblockA{\textit{College of Engineering (of Aff.)} \\
        \textit{Boston University (of Aff.)}\\
        Boston, USA \\
        email address or ORCID}
    \linebreakand
    \IEEEauthorblockN{4\textsuperscript{th} Given Name Surname}
    \IEEEauthorblockA{\textit{College of Engineering (of Aff.)} \\
        \textit{Boston University (of Aff.)}\\
        Boston, USA \\
        email address or ORCID}
    \and
    \IEEEauthorblockN{5\textsuperscript{th} Given Name Surname}
    \IEEEauthorblockA{\textit{College of Engineering (of Aff.)} \\
        \textit{Boston University (of Aff.)}\\
        Boston, USA \\
        email address or ORCID}
}

\maketitle

% Provide a concise summary of the research conducted.
% Include the conclusions reached and the potential implications of those conclusions.
% Your abstract should also:
% 1. consist of a single paragraph up to 250 words, with correct grammar and unambiguous terminology;
% 2. be self-contained with no abbreviations, footnotes, references, or mathematical equations;
% 3. highlight what is unique in your work;
\begin{abstract}
\end{abstract}

% include 3-5 keywords or phrases that describe the research, with any abbreviations clearly defined, to help readers find your paper.
\begin{IEEEkeywords}
\end{IEEEkeywords}

% Help the reader understand why your research is important and what it is contributing to the field.
% Start by giving the reader a brief overview of the current state of research in your subject area.
% Progress to more detailed information on the specific topic of your research.
% End with a description of the exact question or hypothesis that your paper will address.
% Also state your motivation for doing your research and what it will contribute to the field.
\section{Introduction}

% Formulate your research question.
% It should include:
% 1. a detailed description of the question;
% 2. the methods you used to address the question;
% 3. the definitions of any relevant terminology;
% 4. any equations that contributed to your work.
% The methods section should be described in enough detail for someone to replicate your work.
\section{Methods}

% Show the results that you achieved in your work and offer an interpretation of those results.
% Acknowledge any limitations of your work and avoid exaggerating the importance of the results.
\section{Results and Discussion}

% Summarize your key findings.
% Include important conclusions that can be drawn and further implications for the field.
% Discuss benefits or shortcomings of your work and suggest future areas for research.
\section{Conclusion}

% You can recognize individuals who provided assistance with your work, but who do not meet the definition of authorship.
% The acknowledgments section is optional.
\section*{Acknowledgments}

% Provide citation information for all the previous publications referred to in your paper.
% Cite only those references that directly support your work.
\begin{thebibliography}{00}
    \bibitem{b1}
\end{thebibliography}

\vspace{12pt}

\end{document}
